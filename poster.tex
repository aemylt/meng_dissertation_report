% The document class supplies options to control rendering of some standard
% features in the result.  The goal is for uniform style, so some attention 
% to detail is *vital* with all fields.  Each field (i.e., text inside the
% curly braces below, so the MEng text inside {MEng} for instance) should 
% take into account the following:
%
% - author name       should be formatted as "FirstName LastName"
%   (not "Initial LastName" for example),
% - supervisor name   should be formatted as "Title FirstName LastName"
%   (where Title is "Dr." or "Prof." for example),
% - degree programme  should be "BSc", "MEng", "MSci", "MSc" or "PhD",
% - dissertation title should be correctly capitalised (plus you can have
%   an optional sub-title if appropriate, or leave this field blank),
% - dissertation type should be formatted as one of the following:
%   * for the MEng degree programme either "enterprise" or "research" to
%     reflect the stream,
%   * for the MSc  degree programme "$X/Y/Z$" for a project deemed to be
%     X%, Y% and Z% of type I, II and III.
% - year              should be formatted as a 4-digit year of submission
%   (so 2014 rather than the accademic year, say 2013/14 say).
%
% Note there is a *strict* requirement for the poster to be in portrait 
% format so that we display them on the poster boards available.

\documentclass[ % the name of the author
                    author={Dominic Moylett},
                % the name of the supervisor
                supervisor={Dr. Raphael Clifford and Dr. Benjamin Sach},
                % the degree programme
                    degree={MEng},
                % the dissertation    title (which cannot be blank)
                     title={Dictionary Matching with Fingerprints},
                % the dissertation subtitle (which can    be blank)
                  subtitle={},
                % the dissertation     type
                      type={research},
                % the year of submission
                      year={2015} ]{poster}

\newcommand{\nline}{
  \\\vspace{\baselineskip}
}

\begin{document}

% -----------------------------------------------------------------------------

\begin{frame}{} 

\vfill

\begin{columns}[t]
  \begin{column}{0.900\linewidth}
  \begin{block}{\Large Introduction}
  'Big data' is a common term thrown about nowadays. As computers have become more and more powerful, the amount of information we want to process has also grown in size. As a result, time and space efficiency is a significant problem.\nline

  A common method for reducing space is the streaming model, where parts of the input come in at a time instead of the whole input. This is a good choice for pattern matching, where the text can come in one character at a time.\nline

  This project specifically looks at the area of dictionary matching, where you are trying to match one text to many patterns. The Algorithms team have devised a method of solving this problem in $O(\log m)$ time per character and $O(k\log m)$ space, based on Porat and Porat's solution to exact pattern matching in $O(\log m)$ time per character and $O(\log m)$ space. My work is on implementing this algorithm and compare it to other solutions for dictionary matching to see how it performs in practice.
  \end{block}
  \end{column}
\end{columns}

\vfill

\begin{columns}[t]
  \begin{column}{0.422\linewidth}
  \begin{block}{\Large Dictionary Matching Formally}
  We have a text $T$ of $n$ characters and a set $P$ of $k$ patterns $p_1,...,p_k$ with lengths $m_1,...,m_k$. For each index of the text, we output an occurance at $j$ if $\exists i \in \{1,...,k\}$ such that $t_{j - m_i},...,t_j = p_i$.\nline

  The typical solution to dictionary matching in the streaming model is the Aho-Corasick algorithm, which solves the problem based on a generalisation of Knuth-Morris-Pratt. It takes $O(1)$ time per character and $O(\sum_{i =1}^km_i)$ space.
  \end{block}
  \end{column}

  \begin{column}{0.422\linewidth}
  \begin{block}{\Large Pattern Matching in Less Space than the Pattern?}
  Sublinear space can be achieved by using a fingerprint function developed by Karp and Rabin for a string of characters $t_1,...,t_n$, a prime number $p$ and a randomly selected $r$:

  $$\Phi_{p, r}(t_1,...,t_n) = \sum_{i = 1}^{n}r^it_i (\text{mod} p)$$

  These fingerprints compress the amount of space required to store text, and can be modified to match changes in the underlying strings.
  \end{block}
  \end{column}
\end{columns}

\vfill

\begin{columns}[t]
  \begin{column}{0.422\linewidth}
  \begin{block}{\Large Current Progress}
  \begin{itemize}
  \item The Aho-Corasick algorithm has been implemented.
  \item This new algorithm has been implemented for:
    \begin{itemize}
    \item Patterns whose lengths are a power of 2
    \item and patterns which are shorter than $k$
    \end{itemize}
  \item Current work is focusing on long patterns which are repetitive.
  \end{itemize}
  \end{block}
  \end{column}
  \begin{column}{0.422\linewidth}
  \begin{block}{\Large Future Work}
  \begin{itemize}
  \item Finish the work on long patterns which are repetitive.
  \item Implement the algorithm for long patterns.
  \item Compare the performance of the new algorithm to Aho-Corasick.
  \item Implement optimisations to the algorithm.
  \end{itemize}
  \end{block}
  \end{column}
\end{columns}

\vfill

\end{frame}

% -----------------------------------------------------------------------------

\end{document}



