% The document class supplies options to control rendering of some standard
% features in the result.  The goal is for uniform style, so some attention 
% to detail is *vital* with all fields.  Each field (i.e., text inside the
% curly braces below, so the MEng text inside {MEng} for instance) should 
% take into account the following:
%
% - author name       should be formatted as "FirstName LastName"
%   (not "Initial LastName" for example),
% - supervisor name   should be formatted as "Title FirstName LastName"
%   (where Title is "Dr." or "Prof." for example),
% - degree programme  should be "BSc", "MEng", "MSci", "MSc" or "PhD",
% - dissertation title should be correctly capitalised (plus you can have
%   an optional sub-title if appropriate, or leave this field blank),
% - dissertation type should be formatted as one of the following:
%   * for the MEng degree programme either "enterprise" or "research" to
%     reflect the stream,
%   * for the MSc  degree programme "$X/Y/Z$" for a project deemed to be
%     X%, Y% and Z% of type I, II and III.
% - year              should be formatted as a 4-digit year of submission
%   (so 2014 rather than the accademic year, say 2013/14 say).

\documentclass[ % the name of the author
                    author={Dominic Joseph Moylett},
                % the name of the supervisor
                supervisor={Dr. Raphael Clifford and Dr. Benjamin Sach},
                % the degree programme
                    degree={MEng},
                % the dissertation    title (which cannot be blank)
                     title={Dictionary Matching with Fingerprints},
                % the dissertation subtitle (which can    be blank)
                  subtitle={},
                % the dissertation     type
                      type={Research},
                % the year of submission
                      year={2014} ]{dissertation}

\begin{document}

% =============================================================================

% This section simply introduces the structural guidelines.  It can clearly
% be deleted (or commented out) if you use the file as a template for your
% own dissertation: everything following it is in the correct order to use 
% as is.

\iffalse

\section*{Prelude}
\thispagestyle{empty}

A typical dissertation will be structured according to (somewhat) standard 
sections, described in what follows.  However, it is hard and perhaps even 
counter-productive to generalise: the goal is {\em not} to be prescriptive, 
but simply to act as a guideline.  In particular, each page count given is
important but not absolute: their aim is simply to highlight that a clear, 
concise description is better than a rambling alternative that makes it 
hard to separate important content and facts from trivia.

You can use this document as a \LaTeX-based~\cite{latexbook1,latexbook2} 
template for your own dissertation by simply deleting extraneous sections
and content; keep in mind that the associated {\tt Makefile} could be of
use, in particular because it automatically executes \mbox{\BibTeX} to 
deal with the associated bibliography.  

You can, on the other hand, opt {\em not} to use this template; this is a 
perfectly acceptable approach.  Note that a standard cover and declaration 
of authorship may still be produced online via
\[
\mbox{\url{http://www.cs.bris.ac.uk/Teaching/Resources/cover.html}}
\]

\fi

% =============================================================================

% This macro creates the standard UoB title page by using information drawn
% from the document class (meaning it is vital you select the correct degree 
% title and so on).

\maketitle

% After the title page (which is a special case in that it is not numbered)
% comes the front matter or preliminaries; this macro signals the start of
% such content, meaning the pages are numbered with Roman numerals.

\frontmatter

% This macro creates the standard UoB declaration; on the printed hard-copy,
% this must be physically signed by the author in the space indicated.

\makedecl

% LaTeX automatically generates a table of contents, plus associated lists 
% of figures, tables and algorithms.  The former is a compulsory part of the
% dissertation, but if you do not require the latter they can be suppressed
% by simply commenting out the associated macro.

\tableofcontents
\listoffigures
\listoftables
\listofalgorithms
\lstlistoflistings

% The following sections are part of the front matter, but are not generated
% automatically by LaTeX; the use of \chapter* means they are not numbered.

% -----------------------------------------------------------------------------

\chapter*{Executive Summary}

{\bf A compulsory section, of at most $1$ page} 
\vspace{1cm} 

\noindent
This section should pr\'{e}cis the project context, aims and objectives,
and main contributions and achievements; the same section may be called
an abstract elsewhere.  The goal is to ensure the reader is clear about 
what the topic is, what you have done within this topic, {\em and} what 
your view of the outcome is.

The former aspects should be guided by your specification: essentially 
this section is a (very) short version of what is typically the first 
chapter.  The latter aspects should be presented as a concise, factual 
bullet point list.  The points will of course differ for each project, 
but an example is as follows:

\begin{quote}
\noindent
\begin{itemize}
\item I spent $120$ hours collecting material on and learning about the 
      Java garbage-collection sub-system. 
\item I wrote a total of $5000$ lines of source code, comprising a Linux 
      device driver for a robot (in C) and a GUI (in Java) that is 
      used to control it.
\item I designed a new algorithm for computing the non-linear mapping 
      from A-space to B-space using a genetic algorithm, see page $17$.
\item I implemented a version of the algorithm proposed by Jones and 
      Smith in [6], see page $12$, corrected a mistake in it, and 
      compared the results with several alternatives.
\end{itemize}
\end{quote}

% -----------------------------------------------------------------------------
\iffalse
\chapter*{Summary of Changes}

{\bf A conditional section, of at most $1$ page} 
\vspace{1cm} 

Iff. the dissertation represents a resubmission (e.g., as the result of
a resit), this section is compulsory: the content should summarise all
non-trivial changes made to the initial submission.  Otherwise you can
omit it, since a summary of this type is clearly nonsensical.

When included, the section will ideally be used to highlight additional
work completed, and address criticism raised in any associated feedback.
Clearly it is difficult to give generic advice about how to do so, but
an example might be as follows:

\begin{quote}
\noindent
\begin{itemize}
\item Feedback from the initial submission criticised the design and 
      implementation of my genetic algorithm, stating ``there seems 
      to have been no attention to computational complexity during the
      design, and obvious methods of optimisation are missing within
      the resulting implementation''.  Chapter $3$ now includes a
      comprehensive analysis of the algorithm, in terms of both time
      and space.  While I have not altered the algorithm itself, I
      have included a cache mechanism (also detailed in Chapter $3$)
      that provides a significant improvement in average run-time.
\item I added a feature in my implementation to allow automatic rather
      than manual selection of various parameters; the experimental
      results in Chapter $4$ have been updated to reflect this.
\item Questions after the presentation highlighted a range of related
      work that I had not considered: I have make a number of updates 
      to Chapter $2$, resolving this issue.
\end{itemize}
\end{quote}
\fi
% -----------------------------------------------------------------------------

\chapter*{Supporting Technologies}

{\bf A compulsory section, of at most $1$ page}
\vspace{1cm} 

\noindent
This section should present a detailed summary, in bullet point form, 
of any third-party resources (e.g., hardware and software components) 
used during the project.  Use of such resources is always perfectly 
acceptable: the goal of this section is simply to be clear about how
and where they are used, so that a clear assessment of your work can
result.  The content can focus on the project topic itself (rather,
for example, than including ``I used \mbox{\LaTeX} to prepare my 
dissertation''); an example is as follows:

\begin{quote}
\noindent
\begin{itemize}
\item I used the GNU Multiple Precision Arithmetic Library (GMP) to support my implementation of Karp-Rabin fingerprints.
\item I used the C Minimum Perfect Hashing Library (CMPH) for static perfect hashing.
\item I used an open-source implementation of Red-Black Trees from \url{http://en.literateprograms.org/Red-black_tree_(C)?oldid=19567}, with some minor adaptations.
\end{itemize}
\end{quote}

% -----------------------------------------------------------------------------

\chapter*{Notation and Acronyms}

{\bf An optional section, of roughly $1$ or $2$ pages}
\vspace{1cm} 

\noindent
Any well written document will introduce notation and acronyms before
their use, {\em even if} they are standard in some way: this ensures 
any reader can understand the resulting self-contained content.  

Said introduction can exist within the dissertation itself, wherever 
that is appropriate.  For an acronym, this is typically achieved at 
the first point of use via ``Advanced Encryption Standard (AES)'' or 
similar, noting the capitalisation of relevant letters.  However, it 
can be useful to include an additional, dedicated list at the start 
of the dissertation; the advantage of doing so is that you cannot 
mistakenly use an acronym before defining it.  A limited example is 
as follows:

\begin{quote}
\noindent
\begin{tabular}{lcl}
CMPH &: & C Minimum Perfect Hashing Library \\
GMP &: & GNU Multiple Precision Arithmetic Library \\
$T$ &: & A text string of $n$ characters \\
$t_i$ &: & The $i$-th character in T \\
$\mathcal{P}$ &: & A list of $k$ patterns \\
$P_i$ &: & The $i$-th pattern in $\mathcal{P}$, a text string of $m$ characters \\
$p_{i,j}$ &: & The $j$-th character in $P_i$ \\
$\phi(T)$ &: & The Karp-Rabin fingerprint of a text string $T$ \\
\end{tabular}
\end{quote}

% -----------------------------------------------------------------------------

\chapter*{Acknowledgements}

\noindent
First and foremost, I would like to thank my supervisors: Dr. Rapha\"{e}l Clifford and Dr. Benjamin Sach. This project would have been impossible without their work and advice. Alongside them, I would like to mention Dr. Markus Jalsenius for his assistance during the summer project that led to this work and Dr. Allyx Fontaine, who contributed to the paper on which my project is based and advised me alongside Benjamin every week.

Everyone on my course has had an impact on me over the past four years. In particular, I would like to mention William Coaluca, Stephen de Mora, Nicholas Phillips, James Savage and Ashley Whetter. I have put countless hours into many projects with one or more of them.

I would like to acknowledge David Beddows, Derek Bekoe, Timothy Lewis and Jonathan Walsh for remaining a stable household for the past three years - four in the case of David and Timothy.

Last, but most certainly not least, I would like to thank my family for the infinite support, happiness and love they have given me my entire life.

% =============================================================================

% After the front matter comes a number of chapters; under each chapter,
% sections, subsections and even subsubsections are permissible.  The
% pages in this part are numbered with Arabic numerals.  Note that:
%
% - A reference point can be marked using \label{XXX}, and then later
%   referred to via \ref{XXX}; for example Chapter\ref{chap:context}.
% - The chapters are presented here in one file; this can become hard
%   to manage.  An alternative is to save the content in seprate files
%   the use \input{XXX} to import it, which acts like the #include
%   directive in C.

\mainmatter

% -----------------------------------------------------------------------------

\chapter{Contextual Background}
\label{chap:context}

{\bf A compulsory chapter, of roughly $10$ pages}
\vspace{1cm} 

\noindent
This chapter should describe the project context, and motivate each of
the proposed aims and objectives.  Ideally, it is written at a fairly 
high-level, and easily understood by a reader who is technically 
competent but not an expert in the topic itself.

In short, the goal is to answer three questions for the reader.  First, 
what is the project topic, or problem being investigated?  Second, why 
is the topic important, or rather why should the reader care about it?  
For example, why there is a need for this project (e.g., lack of similar 
software or deficiency in existing software), who will benefit from the 
project and in what way (e.g., end-users, or software developers) what 
work does the project build on and why is the selected approach either
important and/or interesting (e.g., fills a gap in literature, applies
results from another field to a new problem).  Finally, what are the 
central challenges involved and why are they significant? 
 
The chapter should conclude with a concise bullet point list that 
summarises the aims and objectives.  For example:

\begin{quote}
\noindent
The high-level objective of this project is to reduce the performance 
gap between hardware and software implementations of modular arithmetic.  
More specifically, the concrete aims are:

\begin{enumerate}
\item Research and survey literature on public-key cryptography and
      identify the state of the art in exponentiation algorithms.
\item Improve the state of the art algorithm so that it can be used
      in an effective and flexible way on constrained devices.
\item Implement a framework for describing exponentiation algorithms
      and populate it with suitable examples from the literature on 
      an ARM7 platform.
\item Use the framework to perform a study of algorithm performance
      in terms of time and space, and show the proposed improvements
      are worthwhile.
\end{enumerate}
\end{quote}

% -----------------------------------------------------------------------------

\chapter{Technical Background}
\label{chap:technical}

\section{Pattern Matching: Formal Definitions}

\noindent
Pattern matching with a single pattern is a simple problem to describe intuitively: We have a text and a pattern, and we want to output any indexes where the pattern occurs in the text.

More formally, we refer to the text by $T$, and define it as a string of $n$ characters $t_0...t_{n-1}$. Likewise, the pattern is referred to as $P$, and is a string of $m$ characters $p_0...p_{m-1}$. The aim of the text indexing problem is to output indexes $i \in \{m-1,...,n-1\}$ such that $t_{i-m+1}...t_{i} = P$.

It is worth noting that there are many other ways of defining this problem. The most notable differences in this paper are that the text and pattern are indexed at zero instead of one, and that the index at the end of the pattern's occurance is returned instead of the index at the start. Both of these are done to be intentionally to be consistent with the code implemented: The zero-indexing is because the implementations are written in C, which also uses zero indexing, and reporting the index at the end of the occurance is to cater for the streaming model, on which more information will be provided later.

\subsection{Dictionary Matching: Formal Definitions}

\noindent
Like pattern matching, dictionary matching is also simple to describe intuitively: We have one text as before, but now we have multiple patterns, and we want to output any indexes where a pattern occurs in the text.

Formally, this is defined as follows: We have a text $n$ characters long $T = t_0...t_{n-1}$, and a set of $k$ patterns $\mathcal{P} = \{P_0,...,P_k\}$ of respective lengths $M = \{m_0,...,m_k\}$. Hence a given pattern $P_i$ is a string of $m_i$ characters $p_{i,0}...p{i,m_i-1}$. We output an index $j \in \{\min(M),...,n_1\}$ if $\exists i \in \{0,...,k-1\}$ such that $t_{j-m_i+1}...t_{j} = P_i$.

Note that for this work, we do not care about what pattern has occured in the text, only that a pattern has occured. This is due to a limitation with one of the algorithms, which will be discussed later.

\section{The Streaming Model}

\noindent
Data streaming is a way of reducing space consumption for certain problems. Under this model, required space is reduced by not processing the entire problem input at once. Instead, the input is provided to the algorithm in portions, delivered via a stream of data. The algorithm processes one portion of the input at a time, and it is required that the algorithm is not allowed to store the entire input.

Under this model, we measure performance by two properties:
\begin{itemize}
  \item {\bf Space:} The size of the data structure
  \item {\bf Time:} The time taken to process each portion in the stream
\end{itemize}

It is easy to see how pattern matching and in turn dictionary matching can be performed in this model. We can process the text by individual characters. During preprocessing we store the pattern and initialise a circular buffer $buf$ which is $m$ characters long. At index $j$ when we receive character $t_j$ we perform the algorithm described in Algorithm~\ref{alg:naive-pattern}. A dictionary matching variant can be done by storing a circular buffer which is $\max(M)$ characters long and repeating Algorithm~\ref{alg:naive-pattern} $k$ times. These algorithms use $O(m)$ and $O(\sum^{k-1}_{i=0}m_i)$ respectively, both in terms of space and time per character.

\begin{algorithm}[t]
{\bf rotate} buf {\bf by one}
{\bf append} $t_i$ {\bf to} buf
\For{$i=0$ {\bf upto} $m-1$}{
  \If{$buf_i \neq p_i$}{
    {\bf return} -1
  }
}
{\bf return} j
\caption{A na\"{i}ve solution to pattern matching.}
\label{alg:naive-pattern}
\end{algorithm}

Of course, these are poor solutions to both pattern and dictionary matching. We can do much better in terms of both time and space complexity.

\section{The Aho-Corasick Algorithm for Dictionary Matching}

The Aho-Corasick Algorithm for Efficient String Matching\cite{Aho:1975:ESM:360825.360855} - known hereafter as Aho-Corasick - is a deterministic algorithm for dictionary matching. Published in 1975, the algorithm works as a generalisation of Knuth-Morris-Pratt, extending the state machine from single patterns in KMP to multiple patterns.

% -----------------------------------------------------------------------------

\chapter{Project Execution}
\label{chap:execution}

{\bf A topic-specific chapter, of roughly $20$ pages} 
\vspace{1cm} 

\noindent
This chapter is intended to describe what you did: the goal is to explain
the main activity or activities, of any type, which constituted your work 
during the project.  The content is highly topic-specific, but for many 
projects it will make sense to split the chapter into two sections: one 
will discuss the design of something (e.g., some hardware or software, or 
an algorithm, or experiment), including any rationale or decisions made, 
and the other will discuss how this design was realised via some form of 
implementation.  

This is, of course, far from ideal for {\em many} project topics.  Some
situations which clearly require a different approach include:

\begin{itemize}
\item In a project where asymptotic analysis of some algorithm is the goal,
      there is no real ``design and implementation'' in a traditional sense
      even though the activity of analysis is clearly within the remit of
      this chapter.
\item In a project where analysis of some results is as major, or a more
      major goal than the implementation that produced them, it might be
      sensible to merge this chapter with the next one: the main activity 
      is such that discussion of the results cannot be viewed separately.
\end{itemize}

\noindent
Note that it is common to include evidence of ``best practice'' project 
management (e.g., use of version control, choice of programming language 
and so on).  Rather than simply a rote list, make sure any such content 
is useful and/or informative in some way: for example, if there was a 
decision to be made then explain the trade-offs and implications 
involved.

\section{Example Section}

This is an example section; 
the following content is auto-generated dummy text.
\lipsum

\subsection{Example Sub-section}

\begin{figure}[t]
\centering
foo
\caption{This is an example figure.}
\label{fig}
\end{figure}

\begin{table}[t]
\centering
\begin{tabular}{|cc|c|}
\hline
foo      & bar      & baz      \\
\hline
$0     $ & $0     $ & $0     $ \\
$1     $ & $1     $ & $1     $ \\
$\vdots$ & $\vdots$ & $\vdots$ \\
$9     $ & $9     $ & $9     $ \\
\hline
\end{tabular}
\caption{This is an example table.}
\label{tab}
\end{table}

\begin{algorithm}[t]
\For{$i=0$ {\bf upto} $n$}{
  $t_i \leftarrow 0$\;
}
\caption{This is an example algorithm.}
\label{alg}
\end{algorithm}

\begin{lstlisting}[float={t},caption={This is an example listing.},label={lst},language=C]
for( i = 0; i < n; i++ ) {
  t[ i ] = 0;
}
\end{lstlisting}

This is an example sub-section;
the following content is auto-generated dummy text.
Notice the examples in Figure~\ref{fig}, Table~\ref{tab}, Algorithm~\ref{alg}
and Listing~\ref{lst}.
\lipsum

\subsubsection{Example Sub-sub-section}

This is an example sub-sub-section;
the following content is auto-generated dummy text.
\lipsum

\paragraph{Example paragraph.}

This is an example paragraph; note the trailing full-stop in the title,
which is intended to ensure it does not run into the text.

% -----------------------------------------------------------------------------

\chapter{Critical Evaluation}
\label{chap:evaluation}

{\bf A topic-specific chapter, of roughly $10$ pages} 
\vspace{1cm} 

\noindent
This chapter is intended to evaluate what you did.  The content is highly 
topic-specific, but for many projects will have flavours of the following:

\begin{enumerate}
\item functional  testing, including analysis and explanation of failure 
      cases,
\item behavioural testing, often including analysis of any results that 
      draw some form of conclusion wrt. the aims and objectives,
      and
\item evaluation of options and decisions within the project, and/or a
      comparison with alternatives.
\end{enumerate}

\noindent
This chapter often acts to differentiate project quality: even if the work
completed is of a high technical quality, critical yet objective evaluation 
and comparison of the outcomes is crucial.  In essence, the reader wants to
learn something, so the worst examples amount to simple statements of fact 
(e.g., ``graph X shows the result is Y''); the best examples are analytical 
and exploratory (e.g., ``graph X shows the result is Y, which means Z; this 
contradicts [1], which may be because I use a different assumption'').  As 
such, both positive {\em and} negative outcomes are valid {\em if} presented 
in a suitable manner.

% -----------------------------------------------------------------------------

\chapter{Conclusion}
\label{chap:conclusion}

{\bf A compulsory chapter, of roughly $2$ pages} 
\vspace{1cm} 

\noindent
The concluding chapter of a dissertation is often underutilised because it 
is too often left too close to the deadline: it is important to allocation
enough attention.  Ideally, the chapter will consist of three parts:

\begin{enumerate}
\item (Re)summarise the main contributions and achievements, in essence
      summing up the content.
\item Clearly state the current project status (e.g., ``X is working, Y 
      is not'') and evaluate what has been achieved with respect to the 
      initial aims and objectives (e.g., ``I completed aim X outlined 
      previously, the evidence for this is within Chapter Y'').  There 
      is no problem including aims which were not completed, but it is 
      important to evaluate and/or justify why this is the case.
\item Outline any open problems or future plans.  Rather than treat this
      only as an exercise in what you {\em could} have done given more 
      time, try to focus on any unexplored options or interesting outcomes
      (e.g., ``my experiment for X gave counter-intuitive results, this 
      could be because Y and would form an interesting area for further 
      study'' or ``users found feature Z of my software difficult to use,
      which is obvious in hindsight but not during at design stage; to 
      resolve this, I could clearly apply the technique of Smith [7]'').
\end{enumerate}

% =============================================================================

% Finally, after the main matter, the back matter is specified.  This is
% typically populated with just the bibliography.  LaTeX deals with these
% in one of two ways, namely
%
% - inline, which roughly means the author specifies entries using the 
%   \bibitem macro and typesets them manually, or
% - using BiBTeX, which means entries are contained in a separate file
%   (which is essentially a databased) then inported; this is the 
%   approach used below, with the databased being dissertation.bib.
%
% Either way, the each entry has a key (or identifier) which can be used
% in the main matter to cite it, e.g., \cite{X}, \cite[Chapter 2}{Y}.

\backmatter

\bibliography{dissertation}

% -----------------------------------------------------------------------------

% The dissertation concludes with a set of (optional) appendicies; these are 
% the same as chapters in a sense, but once signaled as being appendicies via
% the associated macro, LaTeX manages them appropriatly.

\appendix

\chapter{An Example Appendix}
\label{appx:example}

Content which is not central to, but may enhance the dissertation can
be included in one or more appendices; examples include, but are not 
limited to

\begin{itemize}
\item lengthy mathematical proofs, numerical or graphical results
      which are summarised in the main body,
\item sample or example calculations, 
      and
\item results of user studies or questionnaires.
\end{itemize}

\noindent
Note that in line with most research conferences, the marking panel 
is not obliged to read such appendices.

% =============================================================================

\end{document}
